% Created 2021-10-25 Mon 20:29
% Intended LaTeX compiler: pdflatex
\documentclass[presentation,aspectratio=169,8pt]{beamer}
\usepackage[utf8x]{inputenc}
\usepackage[T1]{fontenc}
\usepackage{graphicx}
\usepackage{grffile}
\usepackage{longtable}
\usepackage{wrapfig}
\usepackage{rotating}
\usepackage[normalem]{ulem}
\usepackage{amsmath}
\usepackage{textcomp}
\usepackage{amssymb}
\usepackage{capt-of}
\usepackage{hyperref}
\usepackage{color}
\usepackage[newfloat]{minted}
\usepackage{soul}
\usepackage{mathtools}
\usemintedstyle{tango}
\setminted{mathescape=true}
\setbeamertemplate{itemize items}[circle]
\setbeamertemplate{enumerate items}[default]
\setbeamertemplate{caption}{\raggedright\insertcaption\par}
\setlength{\parskip}{\baselineskip}%
\setlength{\parindent}{0pt}%
\setbeamertemplate{navigation symbols}{}%remove navigation symbols
\newcommand{\hlyellow}[1]{\colorbox{yellow!50}{$\displaystyle#1$}}
\newcommand{\hlfancy}[2]{\sethlcolor{#1}\hl{#2}}
\usetheme{default}
\date{October 29, 2021}
\title{HSGy - 14}
\subtitle{Lenses}
\hypersetup{
 pdfauthor={},
 pdftitle={HSGy - 14},
 pdfkeywords={},
 pdfsubject={},
 pdfcreator={Emacs 28.0.50 (Org mode 9.4.6)}, 
 pdflang={English}}
\begin{document}

\maketitle
\newcommand{\mathcolorbox}[2]{%
  \begingroup
  \setlength{\fboxsep}{2pt}%
  \colorbox{#1}{$\displaystyle #2$}%
  \endgroup
}

\AtBeginSection[]{
  \begin{frame}
  \vfill
  \centering
  \begin{beamercolorbox}[sep=8pt,center,shadow=true,rounded=true]{title}
    \usebeamerfont{title}\insertsectionhead\par%
  \end{beamercolorbox}
  \vfill
  \end{frame}
}

\begin{frame}[label={sec:orge039d3b}]{Haskell Study Group, yo!}
Prerequisites:

\begin{enumerate}
\item You forgot Haskell syntax.
\item You are ready to suffer.
\end{enumerate}
\end{frame}

\begin{frame}[label={sec:org6cb53d4}]{Lenses}
Lenses abstraction make the concept of a field of an abstraction, a first class
notion.

Lenses is a little language of its own.

Used by many interesting libraries. Copycatted to Scala.
\end{frame}

\begin{frame}[label={sec:org24fbdf8},fragile]{Data types}
 \begin{block}{Haskell}
\begin{minted}[]{haskell}
data Atom = Atom { _element :: String, _point :: Point }
data Point = Point { _x :: Double, _y :: Double }
\end{minted}
\end{block}

\begin{block}{Scala}
\begin{minted}[]{scala}
case class Atom(element: String, point: Point)
case class Point(x: Double, y: Double)
\end{minted}
\end{block}
\end{frame}

\begin{frame}[label={sec:org88975d1},fragile]{Getters}
 \begin{columns}
\begin{column}{0.5\columnwidth}
\begin{minted}[]{haskell}
getAtomX :: Atom -> Double
getAtomX = _x . _point
\end{minted}
\end{column}

\begin{column}{0.5\columnwidth}
\begin{minted}[]{scala}
def getAtomX(atom: Atom): Double =
  atom.point.x
\end{minted}
\end{column}
\end{columns}
\end{frame}

\begin{frame}[label={sec:org481528b},fragile]{Setters}
 \begin{columns}
\begin{column}{0.5\columnwidth}
\begin{minted}[]{haskell}
setPoint :: Point -> Atom -> Atom
setPoint p a = a { _point = p }

setElement :: String -> Atom -> Atom
setElement e a = a { _element = e }

setX:: Double -> Point -> Point
setX x p = p { _x = x }

setY:: Double -> Point -> Point
setY y p = p { _y = y }
\end{minted}
\end{column}

\begin{column}{0.5\columnwidth}
\begin{minted}[]{scala}
def setPoint(point: Point, atom: Atom): Atom =
  atom.copy(point = point)

def setElement(element: String, atom: Atom): Atom =
  atom.copy(element = element)

def setX(x: Double, point: Point): Point =
  point.copy(x = x)

def setY(Y: Double, point: Point): Point =
  point.copy(y = y)
\end{minted}
\end{column}
\end{columns}
\end{frame}

\begin{frame}[label={sec:org99a5ffb},fragile]{But what if\ldots{} nested?}
 \begin{columns}
\begin{column}{0.5\columnwidth}
\begin{minted}[]{haskell}
setAtomX :: Double -> Atom -> Atom
setAtomX x a = setPoint (setX x (_point a)) a
\end{minted}
\end{column}

\begin{column}{0.5\columnwidth}
\begin{minted}[]{scala}
def setAtomX(x: Double, atom: Atom): Atom =
  setPoint(setX(x, atom.point), point)
\end{minted}
\end{column}
\end{columns}
\end{frame}

\begin{frame}[label={sec:org77ccaf7}]{Well, that's}
\begin{center}
\includegraphics[height=7.5cm]{images/hory-shet.png}
\end{center}
\end{frame}
\end{document}
