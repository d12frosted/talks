% Created 2022-12-23 Fri 13:46
% Intended LaTeX compiler: pdflatex
\documentclass[presentation,aspectratio=169,smaller]{beamer}
\usepackage[utf8]{inputenc}
\usepackage[T1]{fontenc}
\usepackage{graphicx}
\usepackage{longtable}
\usepackage{wrapfig}
\usepackage{rotating}
\usepackage[normalem]{ulem}
\usepackage{amsmath}
\usepackage{amssymb}
\usepackage{capt-of}
\usepackage{hyperref}
\usepackage{color}
\usepackage[newfloat]{minted}
\usepackage[utf8]{inputenc}
\usepackage{soul}
\usepackage{unicode-math}
\usepackage{mathtools}
\usepackage[mathletters]{ucs}
\usemintedstyle{tango}
\setminted{fontsize=\scriptsize}
\setminted{mathescape=true}
\setbeamertemplate{itemize items}[circle]
\setbeamertemplate{enumerate items}[default]
\setlength{\parskip}{\baselineskip}%
\setlength{\parindent}{0pt}%
\setbeamertemplate{navigation symbols}{}%remove navigation symbols
\newcommand{\hlyellow}[1]{\colorbox{yellow!50}{$\displaystyle#1$}}
\newcommand{\hlfancy}[2]{\sethlcolor{#1}\hl{#2}}
\usetheme{default}
\author{Boris Buliga/Borys Bulyha}
\date{December 23, 2022}
\title{Drinking wine with Emacs}
\subtitle{or a story of technical alcoholism}
\hypersetup{
 pdfauthor={Boris Buliga/Borys Bulyha},
 pdftitle={Drinking wine with Emacs},
 pdfkeywords={},
 pdfsubject={},
 pdfcreator={Emacs 30.0.50 (Org mode 9.6)}, 
 pdflang={English}}
\begin{document}

\maketitle
\newcommand{\mathcolorbox}[2]{%
  \begingroup
  \setlength{\fboxsep}{2pt}%
  \colorbox{#1}{$\displaystyle #2$}%
  \endgroup
}

\AtBeginSection[]{
  \begin{frame}
  \vfill
  \centering
  \begin{beamercolorbox}[sep=8pt,center,shadow=true,rounded=true]{title}
    \usebeamerfont{title}\insertsectionhead\par%
  \end{beamercolorbox}
  \vfill
  \end{frame}
}

\section{Emacs}
\label{sec:org3d9efb7}

\begin{frame}[label={sec:orgd487ae9},fragile]{What is Emacs?\footnote{Daria Tkhorevska asked me to be explicit that no one actually asked this question during Yaniv and Aviran Q\&A session. I made the whole story up.}}
 \begin{columns}
\begin{column}{0.5\columnwidth}
\begin{itemize}
\item \alert{Extensible} text editor.
\item Development began in 1970s.
\item Initially released in 1976.
\item Actively maintained till this day.
\item Use \texttt{C-x C-c} to quit it.
\end{itemize}
\end{column}

\begin{column}{0.5\columnwidth}
\begin{center}
\includegraphics[height=3.5cm]{images/emacs.png}
\end{center}
\end{column}
\end{columns}
\end{frame}

\begin{frame}[label={sec:org1fab026}]{Order salads}
\begin{figure}[htbp]
\centering
\includegraphics[height=5.0cm]{images/salad.png}
\caption{from \texttt{CestDiego/sweetgreen.el}}
\end{figure}
\end{frame}

\begin{frame}[label={sec:orgbdfb417}]{Read email}
\begin{figure}[htbp]
\centering
\includegraphics[height=7.0cm]{images/email-dashboard.png}
\caption{from \texttt{rougier/mu4e-dashboard}}
\end{figure}
\end{frame}

\begin{frame}[label={sec:orgae303f2}]{Control sex toys}
\begin{figure}[htbp]
\centering
\includegraphics[height=5.0cm]{images/deldo.png}
\caption{You may like Emacs, but Kyle (aka Poor Life Choices) \textbf{loves} Emacs. And it's mutual.}
\end{figure}
\end{frame}

\begin{frame}[label={sec:org8057d42}]{Write some code}
\begin{center}
\includegraphics[height=8.0cm]{images/coding.png}
\end{center}
\end{frame}

\begin{frame}[label={sec:org116498e}]{Take notes and manage tasks}
\begin{center}
\includegraphics[height=8.0cm]{images/note-taking.png}
\end{center}
\end{frame}

\begin{frame}[label={sec:org01c0b81}]{Manage your wine cellar}
\begin{center}
\includegraphics[height=8.0cm]{images/wine-notes.png}
\end{center}
\end{frame}

\section*{Intro}
\label{sec:orgb7058e5}
\begin{frame}[label={sec:org0da4cfd}]{About me}
\begin{columns}
\begin{column}{0.75\columnwidth}
\begin{itemize}
\item UA Server Guild manager @Wix.
\item Haskell \(\leftrightarrow\) Emacs Lisp extremist. Whatever that means.
\item Chinese tea lover.
\item Wine-lifestyle activist (92\% of my life).
\end{itemize}
\end{column}

\begin{column}{0.25\columnwidth}
\begin{center}
\includegraphics[height=3.5cm]{images/boris.jpg}
\end{center}
\end{column}
\end{columns}
\end{frame}

\section{I drink wine}
\label{sec:org6f5a0c8}

\begin{frame}[label={sec:org8fe2b30}]{I \sout{love} need\footnote{Is it Obsessive-compulsive disorder?} to take notes}
\begin{enumerate}
\item What, when, where and with whom.
\item Score and tasting notes.
\item Information about wine, producer, region, grape, technology, etc.
\item Manage cellar (inventory?).
\end{enumerate}
\end{frame}

\begin{frame}[label={sec:org8f98e2a}]{What about existing solutions (CellarTracker/Vivino/Delectable)?}
\begin{enumerate}
\item Data is not owned by me.
\item Incorrect data.
\item No way to extend data with extra fields.
\item Capabilities limited by proprietary solution.
\item Requires internet connection (and who is laughing now?).
\item \ldots{}
\item I am engineer, I am capable of creating my own solution.
\end{enumerate}
\end{frame}

\section{Solution}
\label{sec:org4a75de6}

\begin{frame}[label={sec:orgcb72f42}]{Obviously\ldots{}}
\begin{center}
\includegraphics[height=3.5cm]{images/emacs.png}
\end{center}

\begin{quote}
Solution? \sout{Kafka} Emacs.

Problem? You tell me.

--- Kyrylo Ponomariov
\end{quote}
\end{frame}

\begin{frame}[label={sec:org717a093}]{Airtable/Notion}
\begin{figure}[htbp]
\centering
\includegraphics[height=6.0cm]{images/airtable.png}
\caption{Image of my old Tea database (don't have wine there anymore).}
\end{figure}
\end{frame}

\begin{frame}[label={sec:org4f85c59}]{Issues}
\begin{itemize}
\item Too much manual work.
\item Hard to analyse.
\item Hard to write automations (might be outdated).
\item Still not everything is under \emph{my} control.
\item Sometimes blocked by platform (e.g. waiting for features).
\end{itemize}

It's actually good enough. My wife still uses Airtable to manage our home library.
\end{frame}

\begin{frame}[label={sec:org2e71fe0}]{Org Mode for the rescue}
\begin{columns}
\begin{column}{0.75\columnwidth}
\begin{itemize}
\item A markup language (like markdown, but beefed with features).
\item An extension for Emacs.
\item Provides nice APIs to manipulate documents.
\item People use it to write documents, notes, presentations manage tasks and projects, etc.
\end{itemize}
\end{column}

\begin{column}{0.25\columnwidth}
\begin{center}
\includegraphics[height=3.5cm]{images/org-mode-unicorn.png}
\end{center}
\end{column}
\end{columns}
\end{frame}

\begin{frame}[label={sec:org64d8ecd}]{Notes structure}
\begin{center}
\includegraphics[height=7.0cm]{images/notes-structure.png}
\end{center}
\end{frame}

\begin{frame}[label={sec:org1ff6a95}]{Wine entry}
\begin{center}
\includegraphics[height=8.0cm]{images/wine-entry.png}
\end{center}
\end{frame}

\begin{frame}[label={sec:org916a648}]{Rating}
\begin{center}
\includegraphics[height=8.0cm]{images/rating.png}
\end{center}
\end{frame}

\begin{frame}[label={sec:org21ffc73},fragile]{Database}
 \begin{itemize}
\item All notes (e.g. data) are structured.
\item There are APIs to parse these notes.
\item So it's easy to build a database (\texttt{sqlite}) from the notes.
\end{itemize}
\end{frame}

\begin{frame}[label={sec:org7d5fcac}]{Database}
\begin{center}
\includegraphics[height=6.0cm]{images/database.png}
\end{center}
\end{frame}

\begin{frame}[label={sec:orge476030}]{Vulpea}
\begin{columns}
\begin{column}{0.75\columnwidth}
\begin{itemize}
\item A collection of functions for note taking based on Org Mode and Org Roam.
\item A library to write applications and utilities around Org notes (structured or not).
\item Optimized for reads.
\item Allows to query by different metadata, custom fields, links etc.
\item Allows to extend database with custom tables and data extractors.
\item Handles quite big notes collection (10k+).
\item \url{https://github.com/d12frosted/vulpea}
\end{itemize}
\end{column}

\begin{column}{0.25\columnwidth}
\begin{center}
\includegraphics[height=3.5cm]{images/vulpea.png}
\end{center}
\end{column}
\end{columns}
\end{frame}

\begin{frame}[label={sec:orgb1d3991}]{Vino}
\begin{columns}
\begin{column}{0.75\columnwidth}
\begin{itemize}
\item An Emacs application for cellar tracking and wine notes management.
\item \url{https://github.com/d12frosted/vino}
\end{itemize}
\end{column}

\begin{column}{0.25\columnwidth}
\begin{center}
\includegraphics[height=3.5cm]{images/vino.png}
\end{center}
\end{column}
\end{columns}
\end{frame}

\begin{frame}[label={sec:org57fa62c}]{So what?}
\begin{itemize}
\item I have org files as source of truth. These notes are \alert{well-structured}.
\item I have APIs to manipulate these files.
\item I have database with all data extracted (so there is no need to parse these files every time).
\end{itemize}
\end{frame}

\section{Emacs UI is ugly, isn't it? And no one can read your notes.}
\label{sec:orgcef766d}

\begin{frame}[label={sec:org9132fa5}]{barberry.io}
\begin{center}
\includegraphics[height=8.0cm]{images/barberry-garden-home.png}
\end{center}
\end{frame}

\begin{frame}[label={sec:org1fe83b8}]{publicatorg}
\begin{center}
\includegraphics[height=6.0cm]{images/publicatorg.png}
\end{center}
\end{frame}

\begin{frame}[label={sec:orgb4b25d5}]{publicatorg}
\begin{center}
\includegraphics[height=8.0cm]{images/publicatorg-exec.png}
\end{center}
\end{frame}

\begin{frame}[label={sec:org3361d12}]{And it's crazy cool}
\begin{itemize}
\item I can keep my private notes as the source of truth.
\item I have my cozy Emacs UI.
\item But thanks to structured notes and APIs I can build multiple views.
\item \ldots{}
\item Just think about it :) I am overly excited!
\end{itemize}
\end{frame}

\section{Some cool features}
\label{sec:orgfab5220}

\begin{frame}[label={sec:org996143e}]{Code blocks execution}
\begin{columns}
\begin{column}{0.5\columnwidth}
\begin{center}
\includegraphics[height=4.0cm]{images/code-execution-1.png}
\end{center}
\end{column}

\begin{column}{0.5\columnwidth}
\begin{center}
\includegraphics[height=4.0cm]{images/code-execution-2.png}
\end{center}
\end{column}
\end{columns}
\end{frame}

\begin{frame}[label={sec:org4afd33f}]{Fancy on Web and readable in plain text}
\begin{columns}
\begin{column}{0.5\columnwidth}
\begin{center}
\includegraphics[height=5.0cm]{images/graph-1.png}
\end{center}
\end{column}

\begin{column}{0.5\columnwidth}
\begin{center}
\includegraphics[height=3.0cm]{images/graph-2.png}
\end{center}
\end{column}
\end{columns}
\end{frame}

\begin{frame}[label={sec:org8f9735a}]{Convive page}
\begin{center}
\includegraphics[height=7.0cm]{images/convive.png}
\end{center}
\end{frame}

\begin{frame}[label={sec:orgc2e0909}]{There's more}
\begin{itemize}
\item Barberry Garden budget is managed with small Emacs extension based on Vulpea.
\item Plans for tasting events are managed  with small Emacs extension based on Vulpea.
\item I have a view to review my cellar (scattered across 3 places).
\item I have a view that suggests what to post to Vivino (e.g. what was not posted yet).
\item For wine tasting events presentations are generated automatically (no shit) from the event article.
\item I have a simple script that provides me various stats for a given time frame.
\end{itemize}

But there is no time to cover it all, so don't worry.
\end{frame}

\begin{frame}[label={sec:orgcaaccd7}]{Why so complicated?}
\begin{quote}
[…] org-mode is just a collection of lisp running in an editor. It cannot impose more complex features on you. \alert{The genius of org-mode is that you will eventually impose more complex features on yourself.}

--- Michael Hall
\end{quote}
\end{frame}

\begin{frame}[label={sec:org9ebc35d}]{Conclusion}
\begin{itemize}
\item Avoid Emacs at all costs.
\item Structured data is cool.
\item APIs are cool.
\item Combined they can solve lots of routine tasks and result in some interesting products.
\item Don't hesitate to start your own project, even if you are solving your own problems.
\item Drink wine.
\end{itemize}
\end{frame}

\begin{frame}[label={sec:org8c2c63d}]{Seriously, drink wine}
\begin{center}
\includegraphics[height=6cm]{images/tg-barberry-garden.png}
\end{center}
\end{frame}

\begin{frame}[label={sec:orged92127}]{Seriously, drink wine}
\begin{figure}[htbp]
\centering
\includegraphics[height=3.5cm]{images/tg-wix.JPG}
\caption{Wix internal chat}
\end{figure}
\end{frame}

\begin{frame}[label={sec:org45ed7ba}]{Follow}
\begin{itemize}
\item \url{https://barberry.io}
\item \url{https://d12frosted.io}
\item @d12frosted on GitHub
\end{itemize}
\end{frame}

\section{What's next?}
\label{sec:org16a6492}

\begin{frame}[label={sec:orgcd8166a}]{Companion App}
\begin{center}
\includegraphics[height=4.0cm]{images/companion.png}
\end{center}

\begin{itemize}
\item A web application to facilitate wine tasting events.
\begin{itemize}
\item Allow to assign scores, mark favourites and outcasts, etc.
\item Provide technical information about wines.
\item Support blind tasting.
\end{itemize}
\item Wine lovers - find and register for events; see your participation history.
\item Sommeliers - facilitate events and build your audience.
\end{itemize}
\end{frame}

\begin{frame}[label={sec:orgde94e3e}]{Wanna join?}
\begin{itemize}
\item Barberry Garden
\begin{itemize}
\item UX
\item Content writers
\end{itemize}
\item Companion App
\begin{itemize}
\item UX
\item Developers (unfortunately, TypeScript)
\end{itemize}
\item Everything else: there are plenty of ideas, reach me out to discuss
\end{itemize}
\end{frame}

\section{Questions?}
\label{sec:orgd066ea8}

\section{Why not Wix?}
\label{sec:org269d901}

\section{Monetisation?}
\label{sec:org31beebd}

\section{Are you proud?}
\label{sec:org15dc042}

\section{Thank you}
\label{sec:orgbbbfa54}

\begin{frame}[label={sec:org73142f0}]{Links}
\begin{itemize}
\item \url{https://github.com/CestDiego/sweetgreen.el}
\item \url{https://github.com/rougier/mu4e-dashboard}
\item \url{https://github.com/qdot/deldo}
\end{itemize}
\end{frame}
\end{document}